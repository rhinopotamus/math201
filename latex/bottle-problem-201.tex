\documentclass[10pt]{article}
\usepackage[margin=1in, paperwidth=8.5in, paperheight=11in]{geometry}
\usepackage{ifpdf,amsmath, amssymb, comment, color, graphicx, stmaryrd,setspace,enumitem,tikz, fancyhdr, wrapfig, textcomp, units, mathptmx}

\setlength{\headheight}{14.5pt}
\newcommand{\Q}{\mathbb{Q}}
\newcommand{\R}{\mathbb{R}}
\newcommand{\Z}{\mathbb{Z}}

% Solution text is in red. If you want the solutions to show, remove the \iffalse from the definition of the \red command.
\newcommand{\red}[1]{\iffalse \textcolor{red}{#1} \fi}
\newcommand{\blue}[1]{\textcolor{blue}{#1}}
\newcommand{\green}[1]{\textcolor{green}{#1}}
\renewcommand{\section}[1]{\begin{center} \textbf{#1} \\\end{center}}
%
\hyphenpenalty=5000
\setlength{\parindent}{0in}
%\oddsidemargin=-.25in
\allowdisplaybreaks
\pagestyle{fancy}
\renewcommand{\headrulewidth}{0pt}
\lhead{MATH 201}
\rhead{Fall 2025}
%\lfoot{\copyright\ CLEAR Calculus 2010}
\cfoot{}

\begin{document}
%


%\onehalfspacing
\allowdisplaybreaks
%##################################################################
\section{The bottle problem}

Consider plotting height of water in a bottle \textbf{as a function of} the volume of the water in the bottle. That is, height is on the vertical axis (dependent variable) and volume is on the horizontal axis (independent variable).

Throughout, consider a \textbf{fixed} $\Delta V$.

\begin{enumerate}[leftmargin=*]
\item 

% First, let's think about \textbf{functions}, because those are super important for every calculus thing. Explain the meaning of expressing the relationship between height and volume using the function notation \(  h(V) \).

If volume is measured in cups and height is measured in inches, what does the equation \( h(3) = 5\) mean?

\vspace{0.5in}

\item Explore the following claim: Steepness of the graph of the function $h(V)$ has something to do with the cross-sectional area of the bottle.

\vfill
\vfill

\item What bottle shapes could correspond to a straight-line graph? (Be creative!) 
\vfill

\item Consider a pint glass that gets \textbf{wider} as you go up. 

\vfill

\item Consider a volcano vase that gets \textbf{narrower} as you go up.

\vfill

\item What's the deal with a bottle that changes from getting narrower to getting wider? Or one that changes from getting wider to getting narrower?

\vfill

\end{enumerate}
\end{document}