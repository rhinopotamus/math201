\input{../header}

\everymath{\displaystyle}
\begin{document}
%


%\onehalfspacing
\allowdisplaybreaks
%##################################################################
\section{PS12: Definite integrals and the Fundamental Theorem of Calculus}

During a 40-minute workout, a person riding an exercise machine burns calories at a rate of \(c(t)\) calories per minute, where the function \(y = c(t)\) is given by the following information:

\begin{itemize}[leftmargin=0.5cm]
    \item On the interval \(0\leq t\leq 10\), the formula is \(c(t) = -0.05t^2 + t + 10\) (warmup);
    \item on the interval \(10 \leq t \leq 30\), the formula is \(c(t) = 15\) (conditioning phase);
    \item on the interval \(30 \leq t \leq 40\), the formula is \(c(t) = -0.05t^2 + 3t - 30\) (cooldown).
\end{itemize}

Here's a graph of \(c(t)\).

\begin{center}
    \includegraphics[width=\textwidth]{../images/calories-rate-graph.png}
\end{center}

\begin{enumerate}[leftmargin=0.5cm]
    \item (IN1) Shade in the area under \(c(t)\) between \(t = 10\) and \(t = 30\). Use some simple geometry to calculate this area. Give units.
    \item 
    (INx) Write a sentence explaining what the answer to part 1 \textit{means} in the context of the person exercising.
    \item
    (IN2) Use a Riemann sum with 4 rectangles to approximate the area under \(c(t)\) between \(t = 30\) and \(t=40\):
    \begin{itemize}
        \item 
        How wide should each rectangle be?
        \item 
        Decide on a consistent way to choose the height of each rectangle. \\
        (Do you want to do the top-left corner? the top-right corner? the middle? Up to you!)
        \item 
        Sketch your four rectangles on the graph of \(c(t)\).
        \item 
        Compute the four rectangle areas separately, and give units.
        \item 
        Add up the four rectangle areas to get a total area.
        \item 
        Based on your height choices, is your estimate an \textit{over}estimate or an \textit{under}estimate of the actual area? Why?
    \end{itemize}
    \item (IN3) Find a formula for an antiderivative \(C(t)\) for the portion of \(c(t)\) that's on the interval \(0\leq t\leq 10\).
    \item (IN5) Use your antiderivative \(C(t)\) to find the exact value of \(\displaystyle \int_{t=0}^{t=10} c(t)\, dt\). Give units.
    \item (IN5) Now find the \textit{exact} value of \(\displaystyle \int_{t=30}^{t=40} c(t)\, dt\).
    \begin{itemize}
        \item Careful: you'll need to find a new antiderivative, because the formula for \(c(t)\) is different!
    \end{itemize}
    \item Put it all together: Find the \textit{exact} value of \(\displaystyle \int_{t=0}^{t=40} c(t)\, dt\), give units, and explain what this number means in the context of the person exercising.
\end{enumerate}

\end{document}