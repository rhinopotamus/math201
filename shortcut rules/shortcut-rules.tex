\documentclass[letter, 12pt]{article}
\usepackage[margin=1in, paperwidth=8.5in, paperheight=11in]{geometry}
\usepackage{ifpdf,amsmath, amssymb, comment, color, graphicx, stmaryrd,setspace,tikz, fancyhdr, wrapfig, textcomp, units, mathptmx, multicol}
\usepackage[shortlabels]{enumitem}
\hyphenpenalty=5000

\newcommand{\Q}{\mathbb{Q}}
\newcommand{\R}{\mathbb{R}}
\newcommand{\Z}{\mathbb{Z}}

\newcommand{\red}[1]{\textcolor{red}{#1}}
\newcommand{\blue}[1]{\textcolor{blue}{#1}}
\newcommand{\green}[1]{\textcolor{green}{#1}}
\renewcommand{\section}[1]{\begin{center} \textbf{#1} \\\end{center}}
\newcommand{\diff}[1]{\frac{d}{d#1}}
%
\setlength{\parindent}{0in}
%\oddsidemargin=-.25in
\allowdisplaybreaks
\pagestyle{fancy}
\renewcommand{\headrulewidth}{0pt}
\lhead{MATH 201}
\rhead{Fall 2025}
%\lfoot{\copyright\ CLEAR Calculus 2010}
\cfoot{}
\renewcommand{\thefootnote}{*}

\begin{document}
\section{PS6: Derivative shortcut rules}

\textbf{Instructions:} 
In this Problem Set, you'll show how to find derivatives \textbf{step-by-step}, carefully listing out all of your steps. I am being so annoying about making you take very small steps, because it's a good way to get comfortable with all the shortcut rules.

For each problem you choose:
\begin{enumerate}[(a)]
	\item Refer to the list on the next page to identify \textbf{all} the derivative rules you're going to need to use for your problem.
    \item Make your sheet of paper into two columns. 
    \begin{itemize}
        \item In the left column, do tiny derivative steps. 
        \item In the right column, list all the rules you are using in that step.
        \item Look at the example on the next page to see how detailed your steps should be.
    \end{itemize}
	\item Repeat with a new problem until you have used \textbf{all} of the derivative rules in the reference list at least once.
\end{enumerate}
Throughout, carefully label any derivative you find by name, using correct derivative notation.
\vfill
\hrulefill

Here are the problems you can choose from:
\begin{multicols}{2}
\begin{enumerate}[{\bf (1)}]
\item
Find $\dfrac{dy}{dx}$ if \(y=(2x-1)^8-x+e\).
\vskip 1em 
\item 
If \(L(y) = (\tan(y))^{3} + \tan(3)\), find $L'(y)$.
\vskip 1em 
\item
Find $B'(x)$ if \(B(x) = \cos(e^{3 x})\).
\vskip 1em 
\item
Differentiate \(W(x) =\displaystyle{x^4 - 6x + 10 \over \sqrt{7x + 8}}\).
\vskip 1em 
\item
Differentiate \(g(q) = \displaystyle{q^{5} \over \ln(q + {1\over3})}\).
\vskip 1em 
\item
Differentiate \(B(x) = (C e^{3x} - D)(2 x)\), \\
where \(C\) and \(D\) are constants.
\vskip 1em 
\item
Find \(f'(y)\) if \(f(y) = \left(\sqrt{y^4 + 1}\right) \left(\ln\left(\dfrac{y}{2}\right)\right)\).
\item
Find $\dfrac{dB}{dx}$ if 
\(B =\displaystyle{e^{x + \pi} \over \sin(\pi x)}\).
\vskip 1em 
\item 
Differentiate \(h(p) = \displaystyle{(2 p - 1)^{1/4} \over \cos(p + 5)}\).
\vskip 1em 
\item
Differentiate \(f(q)=(\cos q)\left(\sin {\left({{2}\over{\pi}}q\right)}\right)\).
\vskip 1em 
\item
Differentiate \(G = p x^5 + q x^3 - x^r\), \\
where \(p,q\) and \(r\) are constants.
\vskip 1em 
\item
Differentiate \(V = \left(3 x - {1\over4}\right)^{-4} - (5x)^{5/4}\).
\vskip 1em 
\item
Find $\dfrac{dh}{dt}$ if \(h(t) = 7^t \sin(t) + t^7 \cos(t)\).
\vskip 1em 
\item
Differentiate \(h(t) = -3(2t^4 + 7)^4 + 3\).
\end{enumerate}
\end{multicols}
\vfill 

\pagebreak

\textbf{(Example)} Find $\dfrac{dp}{dt}$ if 
$p(t) = \dfrac{\cos(t^2-1)}{e^{-3t}}$.

\vspace{1em}

I am going to need the following derivative rules:
\begin{multicols}{3}
	\begin{itemize}
		\item quotient rule
		\item trig function rule
		\item chain rule
		\item power rule
		\item sum rule
		\item constant function rule
		\item exponential function rule
		\item constant multiple rule
		\item[]
	\end{itemize}
\end{multicols}
Now I will find the derivative, step by step, and 
explain what rules I am using in every step.
I am highlighting changes from the previous line in red.
\begin{align*}
\textbf{Steps}& 
& \textbf{Rules} \\\hline
&&\\
\frac{dp}{dt} &= \frac{[e^{-3t}]\cdot \diff{t}[\cos(t^2-1)] - 
					   [\cos(t^2-1)]\cdot \diff{t}[e^{-3t}]}
					  {[e^{-3t}]^2}
& \textrm{(quotient rule)} \\ \\
&= \frac{[e^{-3t}] \cdot \diff{t}[\cos(t^2-1)] - 
         [\cos(t^2-1)]\cdot
         \red{[e^{-3t}\cdot\diff{t}[-3t]]}}
		{[e^{-3t}]^2} 
& \textrm{(chain rule, } \\ 
&& \textrm{exponential function rule)} \\ \\
&= \frac{[e^{-3t}] \diff{t}[\cos(t^2-1)] - 
         [\cos(t^2-1)]\cdot 
		 [e^{-3t}\cdot\red{(-3)}]}
		{[e^{-3t}]^2} 
& \textrm{(constant multiple rule, } \\
&& \textrm{power rule)} \\ \\
&= \frac{[e^{-3t}] \cdot
         \red{[-\sin(t^2-1)\cdot\diff{t}(t^2-1)]} - 
		 [\cos(t^2-1)]\cdot 
         [e^{-3t}\cdot(-3)]}
		{[e^{-3t}]^2}
& \textrm{(chain rule, } \\
&& \textrm{trig function rule)} \\ \\
&= \frac{[e^{-3t}] \cdot
         [-\sin(t^2-1)\cdot \red{(2t-0)}] - 
		 [\cos(t^2-1)]\cdot 
         [e^{-3t}\cdot(-3)]}
		{[e^{-3t}]^2}
& \textrm{(sum rule, power rule,} \\
&& \textrm{constant function rule)}
\end{align*}

And now I am done, because I have no more ``please take the derivative of'' signs left. Yay!

\pagebreak

\section{List of derivative shortcut rules}

Here is the list of shortcut rules. Check them off as you use them. Keep this list for reference!
{ \everymath{\displaystyle}
    \begin{itemize}[label=\(\square\)]
        \item sum rule: $\diff{x}[f(x)+g(x)] = \diff{x}[f(x)]+\diff{x}[g(x)]$ 
        \item constant multiple rule: $\diff{x}[c\cdot f(x)] = c\cdot\diff{x}[f(x)]$
        \item constant function rule: $\diff{x}[c] = 0$
        \item power rule: $\diff{x}[x^n] = nx^{n-1}$
        \item exponential function rule: $\diff{x}[a^x] = a^x\cdot\ln(a)$
        \item trigonometric function rules: 
        \begin{itemize}[label=\(\bigcirc\)]
            \item $\diff{x}[\sin(x)] = \cos(x)$
            \item $\diff{x}[\cos(x)] = -\sin(x)$
            \item $\diff{x}[\tan(x)] = \sec^2(x)$
        \end{itemize}
        \item product rule: $\diff{x}[f(x) \cdot g(x)] = \diff{x}[f(x)] \cdot g(x) + \diff{x}[g(x)] \cdot f(x)$
        \item quotient rule: $\diff{x}\left[\frac{f(x)}{g(x)}\right] 
        = \frac{g(x) \cdot \diff{x}[f(x)] - f(x) \cdot \diff{x}[g(x)]}{[g(x)]^2}$
        \item chain rule: $[f(g(x))]' = f'(g(x)) \cdot g'(x)$
        \item logarithmic function rule: $\diff{x}[\ln(x)] = \frac{1}{x}$
    \end{itemize}
    \hrulefill

    Just for reference, here are a few more that come up sometimes. They get used much less and they don't show up on this problem set; I would just be sad if I didn't write them down.
    \begin{multicols}{2}
        
        \begin{itemize}[label=\(\centerdot\)]
            \item inverse trigonometric function rules: 
            \begin{itemize}[label=\textopenbullet, series=whale]
                \item $\displaystyle \diff{x}[\arctan(x)] = \frac{1}{1+x^2}$
                \item $\displaystyle \diff{x}[\arcsin(x)] = \frac{1}{\sqrt{1-x^2}}$
            \end{itemize}
            \vfill\null
            \item rules for other trigonometric functions:
            \begin{itemize}[resume*=whale]
                \item $\diff{x}[\cot(x)]=-\csc^2(x)$
                \item $\diff{x}[\sec(x)]=\sec(x)\tan(x)$
                \item $\diff{x}[\csc(x)]=-\csc(x)\cot(x)$
            \end{itemize}
        \end{itemize}
    \end{multicols}
}

\end{document}