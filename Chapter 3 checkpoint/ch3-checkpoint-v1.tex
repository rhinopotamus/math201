\input{../header}
\usepackage{pgfplots}
\rhead{Your name: \rule{8cm}{0.15mm}}

\everymath{\displaystyle}
\begin{document}
%


%\onehalfspacing
\allowdisplaybreaks
%##################################################################
\section{Chapter 3 checkpoint!}

Chapter 1 scorecard:
\begin{center}
    \begin{tabular}{|m{3.75cm}|*{5}{m{2cm}|}} \hline
        Learning target: & DF1 & DF2 & DFa & DFb & AD2 \\\hline
        Your confidence level before starting (0-5): & &&&&\\\hline
        Your confidence level after the quiz (0-5): & &&&&\\\hline
        The mark you earned on this attempt: 
        & Success! \newline Try again!
        & Success! \newline Try again!
        & Success! \newline Try again!
        & Success! \newline Try again!
        & Success! \newline Try again! \\\hline

    \end{tabular}
\end{center}
Chapter 2 scorecard:
\begin{center}
    \begin{tabular}{|m{3.75cm}|*{6}{m{2cm}|}} \hline
        Learning target: & DF3 & DF4 & DF5 & DF6 & DF7 \\\hline
        Your confidence level before starting (0-5): & &&&&\\\hline
        Your confidence level after the quiz (0-5):  & &&&&\\\hline
        The mark you earned on this attempt: 
        & Success! \newline Try again!
        & Success! \newline Try again!
        & Success! \newline Try again!
        & Success! \newline Try again!
        & Success! \newline Try again!\\\hline

    \end{tabular}
\end{center}

Chapter 3 scorecard:
\begin{center}
    \begin{tabular}{|m{3.75cm}|*{6}{m{2cm}|}} \hline
        Learning target: & AD3 & AD4 & AD5 & AD8 & AD9 \\\hline
        Your confidence level before starting (0-5): & &&&&\\\hline
        Your confidence level after the quiz (0-5):  & &&&&\\\hline
        The mark you earned on this attempt: 
        & Success! \newline Try again!
        & Success! \newline Try again!
        & Success! \newline Try again!
        & Success! \newline Try again!
        & Success! \newline Try again!\\\hline

    \end{tabular}
\end{center}

Before anything else, please do the following:
\begin{itemize}
    \item Rank your confidence.
    \item Rip apart the pages.
\end{itemize}

Then do the quiz! Some reminders:
\begin{itemize}
    \item Open notes, closed computer.
    \item If you need more room to write, use the back of the same learning target page, or ask me for some scratch paper.
    \item Read the questions carefully and make sure you're answering each part.
    \item Use good grammar for derivative problems.
    \item Show all your work and explain all your thinking!
\end{itemize}
 

Have fun and do your best! I believe in u $\heartsuit$


%%%%%%%%%%%%%%%%%%%%%%%%%%%%%%%%%%%%%%%%%%%%%%%%%%%%%%%%%
\pagebreak
%%%%%%%%%%%%%%%%%%%%%%%%%%%%%%%%%%%%%%%%%%%%%%%%%%%%%%%%%

\section{Learning target AD3, version 1}

Suppose an oil spill in the ocean formed a circle and was growing at a rate of \(53\) feet\(^2/\)minute. When the oil spill reaches a radius of \(35\) feet, how fast is the radius of the oil spill growing? 

\vspace{1em}

Make sure to do all six steps of the appropriate recipe.

%%%%%%%%%%%%%%%%%%%%%%%%%%%%%%%%%%%%%%%%%%%%%%%%%%%%%%%%%
\pagebreak
%%%%%%%%%%%%%%%%%%%%%%%%%%%%%%%%%%%%%%%%%%%%%%%%%%%%%%%%%

\section{Learning target AD5, version 1}

Consider the function \(f(x)=-2 \, x^{3} + 9 \, x^{2} - 11\).

\begin{enumerate}[leftmargin=0pt]
    \item Make a first-derivative sign chart.
    \vfill

    \item Use the sign chart to decide whether each local extremum is a local minimum or a local maximum.
    \vfill
\end{enumerate}

%%%%%%%%%%%%%%%%%%%%%%%%%%%%%%%%%%%%%%%%%%%%%%%%%%%%%%%%%
\pagebreak
%%%%%%%%%%%%%%%%%%%%%%%%%%%%%%%%%%%%%%%%%%%%%%%%%%%%%%%%%

\section{Learning target AD4, version 1}

Find the absolute minimum and absolute maximum values of the function \(f(x)=-2 \, x^{3} + 9 \, x^{2} - 11\) on the interval \([2,7]\). Show all your work and explain all your thinking.

%%%%%%%%%%%%%%%%%%%%%%%%%%%%%%%%%%%%%%%%%%%%%%%%%%%%%%%%%
\pagebreak
%%%%%%%%%%%%%%%%%%%%%%%%%%%%%%%%%%%%%%%%%%%%%%%%%%%%%%%%%

\section{Learning target AD8, version 1}

You are a dinosaur rancher, and you need to build pens for your three dinosaurs. You decide to build an electric fence around a rectangular region with total area of 25 square kilometers, and then separate this region into three equal-size rectangular pens 
with more fencing. What's the minimum length of electric fencing you need to do this? 

\vspace{1em}

Make sure to do all six steps of the appropriate recipe. Here is one of your three pictures.
\[
	\begin{tabular}{|l|c|c|}
		\hline
 		\includegraphics[width=1in]{../images/t-rex.png} & 
 		\includegraphics[width=1in]{../images/dromiceiomimus.png} &
 		\includegraphics[width=1in]{../images/utahraptor.png} \\
		\hline
		% I need a padding row
		\multicolumn{1}{l}{} & \multicolumn{1}{l}{} & \multicolumn{1}{l}{} 
	\end{tabular}
\]
\vfill



The minimum length of fence is $\underbrace{\qquad\qquad}_{\text{number}}$ $\underbrace{\quad}_{\text{units}}$, 

and the dimensions of the dinosaur pen that make this work are 
$\underbrace{\qquad\qquad}_{\text{number}}\, 
\underbrace{\quad}_{\text{units}}
\times 
\underbrace{\qquad\qquad}_{\text{number}}\,
\underbrace{\quad}_{\text{units}}$.

\iffalse
\section{Learning target AD8, version 1}

A tiger is on the opposite side of a river from a cantaloupe that it wishes to ambush and devour: 

\begin{center}
    \includegraphics[width=0.8\textwidth]{../images/river-tiger-cantaloupe.png}    
\end{center}

The river is 25 feet wide, and the cantaloupe is 120 feet downstream of the tiger. To get to the cantaloupe, the tiger will have to swim diagonally across the river, then get out and run down the bank.
\fi

%%%%%%%%%%%%%%%%%%%%%%%%%%%%%%%%%%%%%%%%%%%%%%%%%%%%%%%%%
\pagebreak
%%%%%%%%%%%%%%%%%%%%%%%%%%%%%%%%%%%%%%%%%%%%%%%%%%%%%%%%%

\section{Learning target AD9, version 1}

For each of the limits below, decide whether it is an indeterminate form. Then, find the value of the limit, applying L'Hopital's rule if appropriate.

\begin{enumerate}[leftmargin=0pt]
    \item \(\displaystyle \lim_{x\to 0 } \frac{ -5 \, \sin\left(7 \, x\right) }{ 3 \, x }\)
    \vfill

    \item \(\displaystyle \lim_{x\to 0 } \frac{ 4 \, \cos\left(8 \, x\right) }{ 2 \, x - 5 }\)
    \vfill
\end{enumerate}



\end{document}