\input{../header}
\usepackage{pgfplots}
\rhead{Your name: \rule{8cm}{0.15mm}}

\everymath{\displaystyle}
\begin{document}
%


%\onehalfspacing
\allowdisplaybreaks
%##################################################################
\section{Chapter 3 checkpoint!}

Chapter 1 scorecard:
\begin{center}
    \begin{tabular}{|m{3.75cm}|*{5}{m{2cm}|}} \hline
        Learning target: & DF1 & DF2 & DFa & DFb & AD2 \\\hline
        Your confidence level before starting (0-5): & &&&&\\\hline
        Your confidence level after the quiz (0-5): & &&&&\\\hline
        The mark you earned on this attempt: 
        & Success! \newline Try again!
        & Success! \newline Try again!
        & Success! \newline Try again!
        & Success! \newline Try again!
        & Success! \newline Try again! \\\hline

    \end{tabular}
\end{center}
Chapter 2 scorecard:
\begin{center}
    \begin{tabular}{|m{3.75cm}|*{6}{m{2cm}|}} \hline
        Learning target: & DF3 & DF4 & DF5 & DF6 & DF7 \\\hline
        Your confidence level before starting (0-5): & &&&&\\\hline
        Your confidence level after the quiz (0-5):  & &&&&\\\hline
        The mark you earned on this attempt: 
        & Success! \newline Try again!
        & Success! \newline Try again!
        & Success! \newline Try again!
        & Success! \newline Try again!
        & Success! \newline Try again!\\\hline

    \end{tabular}
\end{center}

Chapter 3 scorecard:
\begin{center}
    \begin{tabular}{|m{3.75cm}|*{6}{m{2cm}|}} \hline
        Learning target: & AD3 & AD4 & AD5 & AD8 & AD9 \\\hline
        Your confidence level before starting (0-5): & &&&&\\\hline
        Your confidence level after the quiz (0-5):  & &&&&\\\hline
        The mark you earned on this attempt: 
        & Success! \newline Try again!
        & Success! \newline Try again!
        & Success! \newline Try again!
        & Success! \newline Try again!
        & Success! \newline Try again!\\\hline

    \end{tabular}
\end{center}

Before anything else, please do the following:
\begin{itemize}
    \item Rank your confidence.
    \item Rip apart the pages.
\end{itemize}

Then do the quiz! Some reminders:
\begin{itemize}
    \item Open notes, closed computer.
    \item If you need more room to write, use the back of the same learning target page, or ask me for some scratch paper.
    \item Read the questions carefully and make sure you're answering each part.
    \item Use good grammar for derivative problems.
    \item Show all your work and explain all your thinking!
\end{itemize}
 

Have fun and do your best! I believe in u $\heartsuit$


%%%%%%%%%%%%%%%%%%%%%%%%%%%%%%%%%%%%%%%%%%%%%%%%%%%%%%%%%
\pagebreak
%%%%%%%%%%%%%%%%%%%%%%%%%%%%%%%%%%%%%%%%%%%%%%%%%%%%%%%%%

\section{Learning target AD3, version 2}

A wizard is developing a shrinking spell, but he needs to make sure it will work with his pointy hat, which is a cone whose height is three times its radius. Once he casts the spell, the height of his hat shrinks by 2 inches every second. How fast is the volume of his hat changing when the hat is just 3 inches tall? Give units on your answer.

\vspace{1em}

{\tiny Hints: Draw three pictures, label variables and constants, write down what you know and what you want to know, relate the variables, find the derivative wrt time, substitute and solve. 

The volume of a cone is a third of the volume of the cylinder that contains it: $V = 1/3 \cdot \pi r^2 h$.}

%%%%%%%%%%%%%%%%%%%%%%%%%%%%%%%%%%%%%%%%%%%%%%%%%%%%%%%%%
\pagebreak
%%%%%%%%%%%%%%%%%%%%%%%%%%%%%%%%%%%%%%%%%%%%%%%%%%%%%%%%%

\section{Learning target AD5, version 2}

\vspace{-2.5em}

\begin{center}
    {\tiny If I were you, I'd do this one before AD4.}
\end{center}

Consider the function \(g(x) =-2x^{3}+18x^{2}+42x-100\).



\begin{enumerate}[leftmargin=0pt]
    \item Make a first-derivative sign chart.
    
    {\tiny Hint: A first-derivative sign chart is a number line labeled with the critical numbers of $g(x)$, and then the sign of the first derivative on each chunk.}
    \vfill

    \item Use the sign chart to decide whether each local extremum is a local minimum or a local maximum.
    \vfill
\end{enumerate}

%%%%%%%%%%%%%%%%%%%%%%%%%%%%%%%%%%%%%%%%%%%%%%%%%%%%%%%%%
\pagebreak
%%%%%%%%%%%%%%%%%%%%%%%%%%%%%%%%%%%%%%%%%%%%%%%%%%%%%%%%%

\section{Learning target AD4, version 2}

Find the absolute minimum and absolute maximum values of the function \[g(x) =-2x^{3}+18x^{2}+42x-100\] on the interval \([-4, 4]\). Show all your work and explain all your thinking.

%%%%%%%%%%%%%%%%%%%%%%%%%%%%%%%%%%%%%%%%%%%%%%%%%%%%%%%%%
\pagebreak
%%%%%%%%%%%%%%%%%%%%%%%%%%%%%%%%%%%%%%%%%%%%%%%%%%%%%%%%%

\section{Learning target AD8, version 2}

If you take a regular 8.5" x 11" sheet of paper and cut squares out of the corners, you can fold up the flaps on all four sides to make a box (without a lid). How large should you cut the squares so that the resulting box has maximum volume?

{\tiny Hints: Draw three pictures; label variables and constants; write a constraint equation; write an objective function that just has one variable; find any endpoints; do the finding extrema recipe.}

\vfill

The maximum volume is $\underbrace{\qquad\qquad}_{\text{number}}$ $\underbrace{\quad}_{\text{units}}$, 

and the dimensions of the box that make this work are 
$\underbrace{\qquad\qquad}_{\text{number}}\, 
\underbrace{\quad}_{\text{units}}
\times 
\underbrace{\qquad\qquad}_{\text{number}}\,
\underbrace{\quad}_{\text{units}}
\times
\underbrace{\qquad\qquad}_{\text{number}}\,
\underbrace{\quad}_{\text{units}}$.

\iffalse
A tiger is on the opposite side of a river from a cantaloupe that it wishes to ambush and devour: 

\begin{center}
    \includegraphics[width=0.8\textwidth]{../images/river-tiger-cantaloupe.png}    
\end{center}

The river is 25 feet wide, and the cantaloupe is 120 feet downstream of the tiger. To get to the cantaloupe, the tiger will have to swim diagonally across the river, then get out and run down the bank; the time it takes the tiger to run sadfalsdfjlkasdjflkadjsf
\fi

%%%%%%%%%%%%%%%%%%%%%%%%%%%%%%%%%%%%%%%%%%%%%%%%%%%%%%%%%
\pagebreak
%%%%%%%%%%%%%%%%%%%%%%%%%%%%%%%%%%%%%%%%%%%%%%%%%%%%%%%%%

\section{Learning target AD9, version 2} % checkit 18

For each of the limits below, decide whether it is an indeterminate form. Then, find the value of the limit, applying L'Hopital's rule if appropriate. {\tiny Hint: Neither of the answers is 2.}

\begin{enumerate}[leftmargin=0pt]
    \item \(\displaystyle \lim_{x\to 0 }\ \frac{ x^{2} + 4 \, x - 12 }{ x^{2} + 2 \, x - 24 }\)
    \vfill

    \item \(\displaystyle \lim_{x\to -6 }\ \frac{ x^{2} + 4 \, x - 12 }{ x^{2} + 2 \, x - 24 }\)
    \vfill
\end{enumerate}



\end{document}