\documentclass[11 pt]{article}

\setlength{\oddsidemargin}{-.45in}
%\setlength{\evensidemargin}{-.5in}
\setlength{\textwidth}{7.25in}
\setlength{\topmargin}{-0.95in}
\setlength{\textheight}{10.2in}

\usepackage{graphics}
\usepackage{latexsym}
\usepackage{amsfonts}
\usepackage{amsmath}
\usepackage{amsthm}
\usepackage{epstopdf}
%\usepackage{multicol}
%\usepackage{wrapfig}
\pagestyle{empty}

\usepackage[T1]{fontenc}
\usepackage{baskervald}
\usepackage[bigdelims,vvarbb]{newtxmath} 

\newcommand{\ds}{\ensuremath{\displaystyle}}
\newcommand{\tr}{\vspace{0.5in}}
\newcommand{\lr}{\vspace{1.0in}}
\newcommand{\mr}{\vspace{2.0in}}
\newcommand{\br}{\vspace{3.0in}}

\newcommand{\ps}{Problem Set~}
\newcommand{\vi}{\ensuremath{\mathbf{i}}}
\newcommand{\vj}{\ensuremath{\mathbf{j}}}
\newcommand{\vk}{\ensuremath{\mathbf{k}}}
\newcommand{\vr}{\ensuremath{\mathbf{r}}}

\newcommand{\vu}{\ensuremath{\mathbf{u}}}
\newcommand{\va}{\ensuremath{\mathbf{a}}}
\newcommand{\vs}{\ensuremath{\mathbf{s}}}
\newcommand{\ve}{\ensuremath{\mathbf{e}}}
\newcommand{\vx}{\ensuremath{\mathbf{x}}}
\newcommand{\vc}{\ensuremath{\mathbf{c}}}
\newcommand{\vb}{\ensuremath{\mathbf{b}}}

\usepackage{hyperref} 
\hypersetup{colorlinks=true, linkcolor=blue,  anchorcolor=blue,  
citecolor=blue, filecolor=blue, menucolor=blue, pagecolor=blue,  
urlcolor=blue,pdftitle={MTH 201 Syllabus F15}}

\usepackage{framed}

\newcommand{\bdr}{\ensuremath{\mathbb{R}}}

\newcommand{\be}{\begin{enumerate}}
\newcommand{\ee}{\end{enumerate}}

\newcommand{\bi}{\begin{itemize}}  
\newcommand{\ei}{\end{itemize}}

%\renewcommand{\baselinestretch}{1.5}


\begin{document}
 \hfill Boelkins  W25\\
 
\begin{center}
Math 201-06 \\
{\bf Checkpoint 13}
\end{center}


\noindent {\bf Directions:}

\begin{itemize}
	
	\item {\bf Desmos.} We will often use \emph{Desmos} for computations and graphing, just like we do during class.  Any work done in \emph{Desmos} must be summarized and briefly explained in writing on your paper.  
	
	\item {\bf Resources.} During each checkpoint, you may access your 1 page of notes, \emph{Desmos}, and our Blackboard page (for submission); no other uses of your computer or tablet are permitted.  No collaboration with other human beings or use of other internet resources are allowed.
	
	\item {\bf Competency.} Each week, your goal is to demonstrate competency on as many of the included learning targets as you can.  Each question will be marked either ``Y'' or ``NY'': Y for ``yes!'', NY for ``not yet''.  Every learning target will appear on three consecutive checkpoints, so you'll have multiple attempts to demonstrate competency.  ``Competency'' is defined as work that demonstrates clear and complete understanding of the relevant concepts in the learning target. While the work may include a minor error, or one or two minor steps may not be well explained, otherwise the work is complete, correct, and convincing, including the proper use of notation, and appropriate explanation of meaning that uses correct terminology.  (On rare occasions, you might get a ``NY*''; such a mark means you are close to competency and that you have the option to have a short conversation with me prior to the next Checkpoint and attempt to demonstrate competency through the conversation aloud.)	
	
	\item {\bf Submission.}  When finished, use a scanning app to create digital images of your work and upload it to Blackboard under the current Checkpoint location, doing so by 1 pm.
	
\end{itemize}

\noindent \hrulefill \\

\noindent {\bf Learning Targets:} For Checkpoint \#13, there are three learning target being assessed: 

\begin{quote}
{\bf LT 13}:  For an object with a piecewise linear velocity function, I can compute the exact distance traveled and change in position of the object over a given time interval. (Section 4.1)
\end{quote}

\begin{quote}
{\bf LT 14}:  I can compute a Riemann sum and interpret the result as an accumulation of a changing quantity. (Sections 4.1 and 4.2)
\end{quote}

\begin{quote}
{\bf LT 15}:  I can interpret a definite integral in terms of net-signed area.  (Section 4.3)   
\end{quote}

\noindent \hrulefill \\

\vfill \hfill {\bf OVER} 

\pagebreak

%\noindent Name: \\ 

\noindent To demonstrate competency on Learning Target \#$n$, you need to correctly respond to all or nearly all of the prompts in question \#$n$.  If you have already passed a learning target on a prior Checkpoint, you don't need to re-attempt it on this one.
  
  \begin{enumerate}


	\item[13.] An object is moving along a straight path; its velocity is considered positive when moving to the right. The object's velocity $v(t)$ (in meters per second) at time $t$ (in seconds) is given by the graph below. 	

	\begin{figure}[h]
	\begin{center}	
	   \scalebox{0.35}{\includegraphics{images-W25/CP10-velocity.jpg}}
	   \caption{The graph of the velocity function, $v(t)$.}
	\end{center}
	\end{figure}   
	
	\be
	
		\item On the time interval $[2,7]$, determine the object's change in position, including units on your result.  Clearly explain your thinking and process for determining the change in position.
		
		\lr \tr
		
		\item On the time interval $[2,7]$, determine the object's total distance traveled, including units on your result.  Is the distance traveled on this interval the same as the object's change in position?  why or why not? Clearly explain your thinking and process for determining the  distance traveled.
		
		\lr \tr
		
		\item Given an example of an interval $[a,b]$ on which the object's distance traveled is exactly 5 meters (that is, identify numerical values for $a$ and $b$).  Write a sentence to explain and justify your choice of interval.
		
	\ee

\pagebreak

	\item[\#14.] A car brakes hard for a red light, runs slightly into the intersection, and then backs up.  Velocity data for the car is given in the table below.
	
	\begin{center}	
	\begin{tabular}{c|ccccccccc}
	 $t$ (seconds) & 0 & 0.25 & 0.5 & 0.75 & 1 & 1.25 & 1.5 & 1.75 & 2 \\
	 \hline
	 $v(t)$ (feet per second) & 54.5 & 39.2 & 26.5 & 16.2 & 8.5 & 3.2 & 0.5 & -8 & -3 \\ 
	\end{tabular}
\end{center}

	\be
		\item On the time interval $[0,1]$, determine $R_4$, the right Riemann sum for the function $v(t)$ using 4 subintervals.  Clearly show your computations below with proper Riemann sum notation.
		
		\mr
		
		\item In the context of the given information, what is the meaning of $R_4$ that you computed in (a)?  Why?
		
		\lr
		
		\item Now, on the time interval $[0, 2]$, determine $M_4$, the middle Riemann sum for $v(t)$ using 4 subintervals.  Again, clearly show your computations using proper notation.
		
		\mr
		
		\item In the context of the given information, what is the meaning of $M_4$ that you computed in (c)?  Why?
	\ee

\pagebreak

\item[\#15.] Consider the piecewise linear function $y = f(x)$ pictured below.
	
	\begin{center}
		\scalebox{0.45}{\includegraphics{images-W25/CP11-LT14-11.png}}
	\end{center}
	
	Reasoning from the graph, determine the exact value of each of the following definite integrals.  Write at least one sentence to explain your overall thinking and approach.
	
	\be
		\item $\int_{-9}^{-5} f(x) \, dx$
		
		\mr
		
		\item $\int_{-5}^0 f(x) \, dx$
		
		\mr
		
		\item $\int_{-9}^0 f(x) \, dx$ 
	\ee	


  \end{enumerate}


\end{document}
	





