\documentclass[11 pt]{article}

\setlength{\oddsidemargin}{-.45in}
%\setlength{\evensidemargin}{-.5in}
\setlength{\textwidth}{7.25in}
\setlength{\topmargin}{-0.95in}
\setlength{\textheight}{10.2in}

\usepackage{graphics}
\usepackage{latexsym}
\usepackage{amsfonts}
\usepackage{amsmath}
\usepackage{amsthm}
\usepackage{epstopdf}
%\usepackage{multicol}
%\usepackage{wrapfig}
\pagestyle{empty}

\usepackage[T1]{fontenc}
\usepackage{baskervald}
\usepackage[bigdelims,vvarbb]{newtxmath} 

\newcommand{\ds}{\ensuremath{\displaystyle}}
\newcommand{\tr}{\vspace{0.5in}}
\newcommand{\lr}{\vspace{1.0in}}
\newcommand{\mr}{\vspace{2.0in}}
\newcommand{\br}{\vspace{3.0in}}

\newcommand{\ps}{Problem Set~}
\newcommand{\vi}{\ensuremath{\mathbf{i}}}
\newcommand{\vj}{\ensuremath{\mathbf{j}}}
\newcommand{\vk}{\ensuremath{\mathbf{k}}}
\newcommand{\vr}{\ensuremath{\mathbf{r}}}

\newcommand{\vu}{\ensuremath{\mathbf{u}}}
\newcommand{\va}{\ensuremath{\mathbf{a}}}
\newcommand{\vs}{\ensuremath{\mathbf{s}}}
\newcommand{\ve}{\ensuremath{\mathbf{e}}}
\newcommand{\vx}{\ensuremath{\mathbf{x}}}
\newcommand{\vc}{\ensuremath{\mathbf{c}}}
\newcommand{\vb}{\ensuremath{\mathbf{b}}}

\usepackage{hyperref} 
\hypersetup{colorlinks=true, linkcolor=blue,  anchorcolor=blue,  
citecolor=blue, filecolor=blue, menucolor=blue, pagecolor=blue,  
urlcolor=blue,pdftitle={MTH 201 Syllabus F15}}

\usepackage{framed}

\newcommand{\bdr}{\ensuremath{\mathbb{R}}}

\newcommand{\be}{\begin{enumerate}}
\newcommand{\ee}{\end{enumerate}}

\newcommand{\bi}{\begin{itemize}}  
\newcommand{\ei}{\end{itemize}}

%\renewcommand{\baselinestretch}{1.5}


\begin{document}
 \hfill Boelkins  W25\\
 
\begin{center}
Math 201-06 \\
{\bf Checkpoint 1}
\end{center}


\noindent {\bf Directions:}

\begin{itemize}
	
	\item {\bf Desmos.} We will often use \emph{Desmos} for computations and graphing, just like we do during class.  Any work done in \emph{Desmos} must be summarized and briefly explained in writing on your paper.  
	
	\item {\bf Resources.} During each checkpoint, you may access your 1 page of notes, \emph{Desmos}, and our Blackboard page (for submission); no other uses of your computer or tablet are permitted.  No collaboration with other human beings or use of other internet resources are allowed.
	
	\item {\bf Competency.} Each week, your goal is to demonstrate competency on as many of the included learning targets as you can.  Each question will be marked either ``Y'' or ``NY'': Y for ``yes!'', NY for ``not yet''.  Every learning target will appear on three consecutive checkpoints, so you'll have multiple attempts to demonstrate competency.  ``Competency'' is defined as work that demonstrates clear and complete understanding of the relevant concepts in the learning target. While the work may include a minor error, or one or two minor steps may not be well explained, otherwise the work is complete, correct, and convincing, including the proper use of notation, and appropriate explanation of meaning that uses correct terminology.  (On rare occasions, you might get a ``NY*''; such a mark means you are close to competency and that you have the option to have a short conversation with me prior to the next Checkpoint and attempt to demonstrate competency through the conversation aloud.)	
	
	\item {\bf Submission.}  When finished, use a scanning app to create a single PDF of your work and upload it to Blackboard under the current Checkpoint location, doing so by 1 pm.
	
\end{itemize}

\noindent \hrulefill \\

\noindent {\bf Learning Targets:} For Checkpoint \#1, there is one learning target being assessed: 

\begin{quote}
{\bf LT 1}: I can find and interpret average velocity and its units, using information about the position function (as a table, graph, and/or function formula). (Section 1.1) 
\end{quote}

\noindent \hrulefill \\

\vfill \hfill {\bf OVER} 

\pagebreak

\noindent Name: \\ \ \\

\noindent To demonstrate competency on Learning Target \#1, you need to correctly respond to all or nearly all of the prompts in question \#1.  
\begin{enumerate}
	\item[1.]  The position (in miles), $s(t)$, of a car driving along a straight road at time $t$ (in minutes), is given by the following graph. \\
		
	%	\begin{center}
		\scalebox{0.35}{\includegraphics{images/LT1-1-S24.jpg}}
	%	\end{center}
		
	\be
		\item Determine the average velocity of the car between $t=5$ and $t=10$ minutes, $AV_{[5,10]}$. Use proper notation and the work you did to determine the result; include units on your answer.
		
		
		\lr
		
		\item On the graph of $s(t)$ draw a line through  $(3,s(3))$ and $(7,s(7))$; what is the slope of this line and what does that slope mean in the physical context of the function $s$?  
				
		\vspace{0.85in}
		
		
		\item Here's some additional data for the function $s(t)$ that's pictured above: 
	
	\begin{center}
		\begin{tabular}{|c|c|c|c|c|}
		\hline
		$t$ (in minutes) & 8.0 & 8.05 & 8.1 & 8.15 \\
		\hline
		$s(t)$ (in miles) & 4.52254 & 4.54537 & 4.56770 & 4.58952\\
		\hline
		\end{tabular}
		\end{center} 
				
	Find the average velocity of the car on the interval $[8.05, 8.1]$. Label your result using proper notation and include units on your answer.
	
	\lr 
	\ee
	

\end{enumerate}



\end{document}

