\input{../header}
\usepackage{pgfplots}
\rhead{Your name: \rule{8cm}{0.15mm}}

\begin{document}
%


%\onehalfspacing
\allowdisplaybreaks
%##################################################################
\section{Learning target DF0, version 1}

The position (in miles), $s(t)$, of a car driving along a straight road at time $t$ (in minutes), is given by the following graph. \\
		
	%	\begin{center}
		\scalebox{0.35}{\includegraphics{images/LT1-1-S24.jpg}}
	%	\end{center}
		
	\begin{enumerate}
		\item Determine the average velocity of the car between $t=5$ and $t=10$ minutes, $AV_{[5,10]}$. Use proper notation and the work you did to determine the result; include units on your answer.

        \vfill
		\item On the graph of $s(t)$ draw a line through  $(3,s(3))$ and $(7,s(7))$; what is the slope of this line and what does that slope mean in the physical context of the function $s$?  
				
		\vfill
		
		
		\item Here's some additional data for the function $s(t)$ that's pictured above: 
	
	\begin{center}
		\begin{tabular}{|c|c|c|c|c|}
		\hline
		$t$ (in minutes) & 8.0 & 8.05 & 8.1 & 8.15 \\
		\hline
		$s(t)$ (in miles) & 4.52254 & 4.54537 & 4.56770 & 4.58952\\
		\hline
		\end{tabular}
		\end{center} 
				
	Find the average velocity of the car on the interval $[8.05, 8.1]$. Label your result using proper notation and include units on your answer.
    \vfill
    \end{enumerate}


%%%%%%%%%%%%%%%%%%%%%%%%%%%%%%%%%%%%%%%%%%%%%%%%%%%%%%%%%
\pagebreak
%%%%%%%%%%%%%%%%%%%%%%%%%%%%%%%%%%%%%%%%%%%%%%%%%%%%%%%%%
\section{Learning targets DF1 and DFa, version 2}

Arapaho Glacier is a mountain glacier in Roosevelt National Forest,
west of Boulder, CO. The following table\footnote{
Haugen, B., Scambos, T., Pfeffer, T., \& Anderson, R. (2010). Twentieth-century changes in the thickness and extent of Arapaho Glacier, Front Range, Colorado. \textit{Arctic, Antarctic, and Alpine Research, 42}(2), 198-209.}
gives the surface area, $A(t)$, in square meters, of Arapaho Glacier in the year $t$. 
\begin{center}
	\begin{tabular}{|c|c|c|c|c|}
	\hline
	$t$ & 1900 & 1960 & 1973 & 1999\\
	\hline
	$A(t)$ & 338,282 & 250,764 & 225,000 & 162,027\\
	\hline
	\end{tabular}
 \end{center}
	
\begin{enumerate}[leftmargin=0pt]

\item Compute an approximation for $A'(1960)$, and {\bf include units} for this number.\\
Write a sentence explaining what the number means about how the area of the glacier is changing.\\
\textbf{Don't say ``per,'' and don't say ``rate.''}

\vfill
\vfill

\item Compute an approximation for $A'(1999)$, and {\bf include units} for this number.\\
Do you think your approximation is too high or too low? Why?

\vfill
\vfill

\item How does $A'(1960)$ compare to $A'(1999)$? Is that good or bad?
\vfill
\end{enumerate}
%%%%%%%%%%%%%%%%%%%%%%%%%%%%%%%%%%%%%%%%%%%%%%%%%%%%%%%%%
\pagebreak
%%%%%%%%%%%%%%%%%%%%%%%%%%%%%%%%%%%%%%%%%%%%%%%%%%%%%%%%%

\section{Learning target DF2, version 1}

Suppose that $f(x) = 3x^2 - 5x + 4$.
\begin{enumerate}[leftmargin=0pt]
    \item Use the limit definition of the derivative to find $f'(x)$. \textbf{No shortcut rules!}
    
    \vfill
    \vfill

    \item Evaluate at $x=8$.

    \vfill
    \item (Bonus!) What happens to the 3? What about the $-5$? And the $4$?
\end{enumerate}
%%%%%%%%%%%%%%%%%%%%%%%%%%%%%%%%%%%%%%%%%%%%%%%%%%%%%%%%%
\pagebreak
%%%%%%%%%%%%%%%%%%%%%%%%%%%%%%%%%%%%%%%%%%%%%%%%%%%%%%%%%

\section{Learning target DF2, version 2}

Suppose that  \(g(w) = 6 \, w^{3} - 2 \, w^{2} - 9 \, w - 2\). Use the limit definition of the derivative to find $g'(w)$.

\vspace{1em} 

Algebra hint: $(w+h)^3 = w^3 + 3w^2 h + 3w h^2 + h^3$.
\vfill
PS - The answer is $g'(w) = 18w^2 - 4w + 9.$


%%%%%%%%%%%%%%%%%%%%%%%%%%%%%%%%%%%%%%%%%%%%%%%%%%%%%%%%%
\pagebreak
%%%%%%%%%%%%%%%%%%%%%%%%%%%%%%%%%%%%%%%%%%%%%%%%%%%%%%%%%

\section{Learning target DFb, version 2}

Here is the graph of some wacky function $h(t)$:

\begin{center}
    \includegraphics[width=0.9\textwidth]{../images/DFb-v2.png}    
\end{center}


Sketch the graph of $h'(t)$ on the blank axes below.

\begin{center}
    \includegraphics[width=0.9\textwidth]{../images/DFb-v2-blank.png}    
\end{center}

\end{document}