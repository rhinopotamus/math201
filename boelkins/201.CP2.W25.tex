\documentclass[11 pt]{article}

\setlength{\oddsidemargin}{-.45in}
%\setlength{\evensidemargin}{-.5in}
\setlength{\textwidth}{7.25in}
\setlength{\topmargin}{-0.95in}
\setlength{\textheight}{10.2in}

\usepackage{graphics}
\usepackage{latexsym}
\usepackage{amsfonts}
\usepackage{amsmath}
\usepackage{amsthm}
\usepackage{epstopdf}
%\usepackage{multicol}
%\usepackage{wrapfig}
\pagestyle{empty}

\usepackage[T1]{fontenc}
\usepackage{baskervald}
\usepackage[bigdelims,vvarbb]{newtxmath} 

\newcommand{\ds}{\ensuremath{\displaystyle}}
\newcommand{\tr}{\vspace{0.5in}}
\newcommand{\lr}{\vspace{1.0in}}
\newcommand{\mr}{\vspace{2.0in}}
\newcommand{\br}{\vspace{3.0in}}

\newcommand{\ps}{Problem Set~}
\newcommand{\vi}{\ensuremath{\mathbf{i}}}
\newcommand{\vj}{\ensuremath{\mathbf{j}}}
\newcommand{\vk}{\ensuremath{\mathbf{k}}}
\newcommand{\vr}{\ensuremath{\mathbf{r}}}

\newcommand{\vu}{\ensuremath{\mathbf{u}}}
\newcommand{\va}{\ensuremath{\mathbf{a}}}
\newcommand{\vs}{\ensuremath{\mathbf{s}}}
\newcommand{\ve}{\ensuremath{\mathbf{e}}}
\newcommand{\vx}{\ensuremath{\mathbf{x}}}
\newcommand{\vc}{\ensuremath{\mathbf{c}}}
\newcommand{\vb}{\ensuremath{\mathbf{b}}}

\usepackage{hyperref} 
\hypersetup{colorlinks=true, linkcolor=blue,  anchorcolor=blue,  
citecolor=blue, filecolor=blue, menucolor=blue, pagecolor=blue,  
urlcolor=blue,pdftitle={MTH 201 Syllabus F15}}

\usepackage{framed}

\newcommand{\bdr}{\ensuremath{\mathbb{R}}}

\newcommand{\be}{\begin{enumerate}}
\newcommand{\ee}{\end{enumerate}}

\newcommand{\bi}{\begin{itemize}}  
\newcommand{\ei}{\end{itemize}}

%\renewcommand{\baselinestretch}{1.5}


\begin{document}
 \hfill Boelkins  W25\\
 
\begin{center}
Math 201-06 \\
{\bf Checkpoint 2}
\end{center}


\noindent {\bf Directions:}

\begin{itemize}
	
	\item {\bf Desmos.} We will often use \emph{Desmos} for computations and graphing, just like we do during class.  Any work done in \emph{Desmos} must be summarized and briefly explained in writing on your paper.  
	
	\item {\bf Resources.} During each checkpoint, you may access your 1 page of notes, \emph{Desmos}, and our Blackboard page (for submission); no other uses of your computer or tablet are permitted.  No collaboration with other human beings or use of other internet resources are allowed.
	
	\item {\bf Competency.} Each week, your goal is to demonstrate competency on as many of the included learning targets as you can.  Each question will be marked either ``Y'' or ``NY'': Y for ``yes!'', NY for ``not yet''.  Every learning target will appear on three consecutive checkpoints, so you'll have multiple attempts to demonstrate competency.  ``Competency'' is defined as work that demonstrates clear and complete understanding of the relevant concepts in the learning target. While the work may include a minor error, or one or two minor steps may not be well explained, otherwise the work is complete, correct, and convincing, including the proper use of notation, and appropriate explanation of meaning that uses correct terminology.  (On rare occasions, you might get a ``NY*''; such a mark means you are close to competency and that you have the option to have a short conversation with me prior to the next Checkpoint and attempt to demonstrate competency through the conversation aloud.)	
	
	\item {\bf Submission.}  When finished, use a scanning app to create a single PDF of your work and upload it to Blackboard under the current Checkpoint location, doing so by 1 pm.
	
\end{itemize}

\noindent \hrulefill \\

\noindent {\bf Learning Targets:} For Checkpoint \#2, there are four learning target being assessed: 

\begin{quote}
{\bf LT 1}: I can find and interpret average velocity and its units, using information about the position function (as a table, graph, and/or function formula). (Section 1.1) 
\end{quote}

\begin{quote}
{\bf LT 2}: I can estimate the derivative of a function at a given value using numerical or graphical data. 
(Section 1.3)
\end{quote}

\begin{quote}
{\bf LT 3}: I can use the limit definition of the derivative to find the derivative function. (Section 1.4) 
\end{quote}

\begin{quote}
{\bf LT 4}: Given a graph of a function, I can sketch an accurate graph of the derivative (including the correct sign, direction, relative heights, zeros, and any places where the derivative doesn’t exist). (Section 1.4) 
\end{quote}

\noindent \hrulefill \\

\vfill \hfill {\bf OVER} 

\pagebreak

%\noindent Name: \\ 

\noindent To demonstrate competency on Learning Target \#n, you need to correctly respond to all or nearly all of the prompts in question \#n.  If you have already passed a learning target on a prior Checkpoint, you don't need to re-attempt it on this one.
  
\begin{enumerate}

	\item[1.] Respond to both parts (a) and (b) below regarding average velocity.
	\be

		\item Your fitbit odometer records some data about the distance you traveled on a bike ride around Reeds Lake in Grand Rapids. This data is recorded in the table below.
	
	\begin{center}
		\begin{tabular}{|c|c|c|c|c|c|c|}
		\hline
		$t$ (in minutes) & 0 & 15 & 30 & 45 & 60 & 75 \\
		\hline
		$d(t)$ (in miles) & 0 & 4.5 & 10.5 & 15 & 21 & 24\\
		\hline
		\end{tabular}
		\end{center}
	Find your average velocity over the time interval from $t = 30$ to $t = 75$. Label your result using proper notation and include units on your answer.
	
	\lr 	
	
		\item The following graph shows a plot of the function $s(t)$ that measures the location of a car (in thousands of feet) along a road at time $t$ in minutes.
		
		\begin{center}
		\scalebox{0.65}{\includegraphics{images/CP-3-01-s.png}}
		\end{center}
		
		\begin{itemize}
		\item Use the graph to find the average velocity of the car between $t=2$ and $t=7$ minutes, $AV_{[2,7]}$. Include units on your answer.
		
		
		\lr
		
		\item On the graph of $s(t)$ draw a line through $(7,s(7))$ and $(10,s(10))$; what is the slope of this line and what does that slope mean in the physical context of the function $s$?  Write to explain, being careful and precise.
		
		\tr
		
		
		\end{itemize}
		
		
		

	\ee
	
	\vfill \hfill {\bf OVER}
	
	\pagebreak
	
%\noindent Name: \\ 
	
	\item[2.] 
	\be
	
		\item Suppose we know the following data for a function $g$:

\begin{center}  
  \begin{tabular}{r|ccccccc}
  	$t$ & -8 & -5 & -2 & 1 & 4 & 7 & 10 \\
	\hline
	$g(t)$ & 21 & 20 & 17 & 12 & 7 & 4 & 3 \\
  \end{tabular}
\end{center}

Estimate $g'(-5)$ and $g'(4)$.  Clearly show your work and thinking, using proper notation.

\lr \tr
	
		\item  For the function $f$ whose graph is given below, use the graph to provide accurate estimates of \\
		 $f'(-2)$, $f'(0.25)$, and $f'(4)$.  Clearly state your results below the graph using clear notation like \\ ``$f'(-2) \approx$ \underline{\hspace{0.25in}}''; write one sentence to explain your thinking.


\ \hspace{0.5in}  \scalebox{0.5}{\includegraphics{images/CP3-01-ff'.png}} 


	\ee


	\vfill \hfill {\bf OVER}
	
	\pagebreak	

%\noindent Name: \\ 
	
	\item[3.] Let $\displaystyle p(x) = 4 - 3x - x^2$. \\ \ \\
	 Use the limit definition of the derivative to determine a formula for $p'(x)$.  Clearly show all of your steps using proper notation.
	
	
\br \mr
	
	\item[4.] For the function $f$ plotted at left, sketch an accurate graph of its derivative, $f'$, on the axes at right.  Write at least one sentence below your graph to explain the process/thinking by which you developed your graph of $f'$. \\  

\begin{center}  
  \scalebox{0.39}{\includegraphics{images/CP4-01-ff'.png}} \hspace{0.2in} \scalebox{0.39}{\includegraphics{images/CP4-01-blank.png}} \\
\end{center}


\end{enumerate}


\end{document}
