\documentclass[11 pt]{article}

\setlength{\oddsidemargin}{-.45in}
%\setlength{\evensidemargin}{-.5in}
\setlength{\textwidth}{7.25in}
\setlength{\topmargin}{-0.95in}
\setlength{\textheight}{10.2in}

\usepackage{graphics}
\usepackage{latexsym}
\usepackage{amsfonts}
\usepackage{amsmath}
\usepackage{amsthm}
\usepackage{epstopdf}
%\usepackage{multicol}
%\usepackage{wrapfig}
\pagestyle{empty}

\usepackage[T1]{fontenc}
\usepackage{baskervald}
\usepackage[bigdelims,vvarbb]{newtxmath} 

\newcommand{\ds}{\ensuremath{\displaystyle}}
\newcommand{\tr}{\vspace{0.5in}}
\newcommand{\lr}{\vspace{1.0in}}
\newcommand{\mr}{\vspace{2.0in}}
\newcommand{\br}{\vspace{3.0in}}

\newcommand{\ps}{Problem Set~}
\newcommand{\vi}{\ensuremath{\mathbf{i}}}
\newcommand{\vj}{\ensuremath{\mathbf{j}}}
\newcommand{\vk}{\ensuremath{\mathbf{k}}}
\newcommand{\vr}{\ensuremath{\mathbf{r}}}

\newcommand{\vu}{\ensuremath{\mathbf{u}}}
\newcommand{\va}{\ensuremath{\mathbf{a}}}
\newcommand{\vs}{\ensuremath{\mathbf{s}}}
\newcommand{\ve}{\ensuremath{\mathbf{e}}}
\newcommand{\vx}{\ensuremath{\mathbf{x}}}
\newcommand{\vc}{\ensuremath{\mathbf{c}}}
\newcommand{\vb}{\ensuremath{\mathbf{b}}}

\usepackage{hyperref} 
\hypersetup{colorlinks=true, linkcolor=blue,  anchorcolor=blue,  
citecolor=blue, filecolor=blue, menucolor=blue, pagecolor=blue,  
urlcolor=blue,pdftitle={MTH 201 Syllabus F15}}

\usepackage{framed}

\newcommand{\bdr}{\ensuremath{\mathbb{R}}}

\newcommand{\be}{\begin{enumerate}}
\newcommand{\ee}{\end{enumerate}}

\newcommand{\bi}{\begin{itemize}}  
\newcommand{\ei}{\end{itemize}}

%\renewcommand{\baselinestretch}{1.5}


\begin{document}
 \hfill Boelkins  W25\\
 
\begin{center}
Math 201-06 \\
{\bf Checkpoint 9}
\end{center}


\noindent {\bf Directions:}

\begin{itemize}
	
	\item {\bf Desmos.} We will often use \emph{Desmos} for computations and graphing, just like we do during class.  Any work done in \emph{Desmos} must be summarized and briefly explained in writing on your paper.  
	
	\item {\bf Resources.} During each checkpoint, you may access your 1 page of notes, \emph{Desmos}, and our Blackboard page (for submission); no other uses of your computer or tablet are permitted.  No collaboration with other human beings or use of other internet resources are allowed.
	
	\item {\bf Competency.} Each week, your goal is to demonstrate competency on as many of the included learning targets as you can.  Each question will be marked either ``Y'' or ``NY'': Y for ``yes!'', NY for ``not yet''.  Every learning target will appear on three consecutive checkpoints, so you'll have multiple attempts to demonstrate competency.  ``Competency'' is defined as work that demonstrates clear and complete understanding of the relevant concepts in the learning target. While the work may include a minor error, or one or two minor steps may not be well explained, otherwise the work is complete, correct, and convincing, including the proper use of notation, and appropriate explanation of meaning that uses correct terminology.  (On rare occasions, you might get a ``NY*''; such a mark means you are close to competency and that you have the option to have a short conversation with me prior to the next Checkpoint and attempt to demonstrate competency through the conversation aloud.)	
	
	\item {\bf Submission.}  When finished, use a scanning app to create digital images of your work and upload it to Blackboard under the current Checkpoint location, doing so by 1 pm.
	
\end{itemize}

\noindent \hrulefill \\

\noindent {\bf Learning Targets:} For Checkpoint \#9, there are three learning target being assessed: 

\begin{quote}
{\bf LT 9}: I can use the chain rule to find the derivative of a composite function. (Section 2.5) 
\end{quote}

\begin{quote}
{\bf LT 10}:  I can use implicit differentiation to find $\frac{dy}{dx}$ for a function defined implicitly, using basic derivative and structure rules in the process.  (Section 2.7)
\end{quote}

\begin{quote}
{\bf LT 11}:  I can solve related rates problems and interpret the results in the given context. (Section 3.5)
\end{quote}


\noindent \hrulefill \\

\vfill \hfill {\bf OVER} 

\pagebreak

%\noindent Name: \\ 

\noindent To demonstrate competency on Learning Target \#$n$, you need to correctly respond to all or nearly all of the prompts in question \#$n$.  If you have already passed a learning target on a prior Checkpoint, you don't need to re-attempt it on this one.
  
  \begin{enumerate}

	\item[9.]  Consider the two piecewise linear functions $g$ and $h$ given by the figure below.
	
 
\scalebox{0.4}{\includegraphics{images-W25/HW10-S24-PW.jpg}} \\


Determine each of the following derivative values that involve functions that are defined in terms of $g$ and $h$.  Clearly show your work and thinking using proper notation, be sure your values are \emph{exact} in light of the given information, and write one sentence to explain your overall approach.

	\be
		\item Let $M(x) = h(g(x))$.  Determine $H'(3)$.  Write your work and conclusion in the space to the right of the graph above.
		
		\item Let $N(x) = g(x^3)$.  Determine $N'(-1)$.
		
		\mr
		
		\item Let $P(x) = \ln(h(x))$.  Determine $P'(1)$.
	\ee

	\vfill \hfill {\bf OVER}

\pagebreak

	\item[10.]  Consider the curve defined implicitly by the equation
	$$xy^3 + 2xy = 6$$
	
	\be
	
		\item Use implicit differentiation to find an expression for $\frac{dy}{dx}$ that depends on $x$ and $y$.  Show all of your work and thinking clearly, using correct notation.
		
		\br \lr
		
		\item Notice that the point $(2,1)$ lies on the curve $xy^3 + 2xy = 6$.  Find the exact slope of the tangent line to curve at $(2,1)$.  Show clearly how you determined this value, and label your work with proper notation.
	
	\ee

\pagebreak

	\item[11.]   Gravel is being dumped from a conveyor belt at a rate of 10 cubic feet per minute. The gravel forms a pile in the shape of a right circular cone whose base diameter and height are always the same. 
	
	At what instantaneous rate is the height of the pile increasing when the pile is 12 ft high? 
	
	
% A rocket is launched and a laser range finder follows the path of the rocket.  The range finder is located on the ground 300 feet from the point where the rocket was launched.  The rocket rises vertically and does so at a constant rate of 20 feet per second.
	
% At what instantaneous rate is the distance between the laser range finder and the rocket changing at the instant the laser range finder reads ``500 feet''.

%Consider a triangle with base $b$ and height $h$ where both the base and height are changing as time elapses.  Suppose we know the following information: the height of the triangle is always 5 times the length of the base, and at the instant the height is $h = 9$ inches, the height is increasing at an instantaneous rate of $0.48$ inches per second.  
	
%	How fast is the triangle's area increasing at this same instant?

\emph{	To earn a Y on this question, you must fully justify your answer using calculus, including 
	\begin{itemize}
		\item a clear, labeled diagram with variables identified
		\item appropriate equations that relate changing quantities and their rates of change
		\item proper notation involving derivatives
		\item a clear summary of your result, with correct units
	\end{itemize}}




  \end{enumerate}


\end{document}
	





