\documentclass[11 pt]{article}

\setlength{\oddsidemargin}{-.45in}
%\setlength{\evensidemargin}{-.5in}
\setlength{\textwidth}{7.25in}
\setlength{\topmargin}{-0.95in}
\setlength{\textheight}{10.2in}

\usepackage{graphics}
\usepackage{latexsym}
\usepackage{amsfonts}
\usepackage{amsmath}
\usepackage{amsthm}
\usepackage{epstopdf}
%\usepackage{multicol}
%\usepackage{wrapfig}
\pagestyle{empty}

\usepackage[T1]{fontenc}
\usepackage{baskervald}
\usepackage[bigdelims,vvarbb]{newtxmath} 

\newcommand{\ds}{\ensuremath{\displaystyle}}
\newcommand{\tr}{\vspace{0.5in}}
\newcommand{\lr}{\vspace{1.0in}}
\newcommand{\mr}{\vspace{2.0in}}
\newcommand{\br}{\vspace{3.0in}}

\newcommand{\ps}{Problem Set~}
\newcommand{\vi}{\ensuremath{\mathbf{i}}}
\newcommand{\vj}{\ensuremath{\mathbf{j}}}
\newcommand{\vk}{\ensuremath{\mathbf{k}}}
\newcommand{\vr}{\ensuremath{\mathbf{r}}}

\newcommand{\vu}{\ensuremath{\mathbf{u}}}
\newcommand{\va}{\ensuremath{\mathbf{a}}}
\newcommand{\vs}{\ensuremath{\mathbf{s}}}
\newcommand{\ve}{\ensuremath{\mathbf{e}}}
\newcommand{\vx}{\ensuremath{\mathbf{x}}}
\newcommand{\vc}{\ensuremath{\mathbf{c}}}
\newcommand{\vb}{\ensuremath{\mathbf{b}}}

\usepackage{hyperref} 
\hypersetup{colorlinks=true, linkcolor=blue,  anchorcolor=blue,  
citecolor=blue, filecolor=blue, menucolor=blue, pagecolor=blue,  
urlcolor=blue,pdftitle={MTH 201 Syllabus F15}}

\usepackage{framed}

\newcommand{\bdr}{\ensuremath{\mathbb{R}}}

\newcommand{\be}{\begin{enumerate}}
\newcommand{\ee}{\end{enumerate}}

\newcommand{\bi}{\begin{itemize}}  
\newcommand{\ei}{\end{itemize}}

%\renewcommand{\baselinestretch}{1.5}


\begin{document}
 \hfill Boelkins  W25\\
 
\begin{center}
Math 201-06 \\
{\bf Checkpoint 5}
\end{center}


\noindent {\bf Directions:}

\begin{itemize}
	
	\item {\bf Desmos.} We will often use \emph{Desmos} for computations and graphing, just like we do during class.  Any work done in \emph{Desmos} must be summarized and briefly explained in writing on your paper.  
	
	\item {\bf Resources.} During each checkpoint, you may access your 1 page of notes, \emph{Desmos}, and our Blackboard page (for submission); no other uses of your computer or tablet are permitted.  No collaboration with other human beings or use of other internet resources are allowed.
	
	\item {\bf Competency.} Each week, your goal is to demonstrate competency on as many of the included learning targets as you can.  Each question will be marked either ``Y'' or ``NY'': Y for ``yes!'', NY for ``not yet''.  Every learning target will appear on three consecutive checkpoints, so you'll have multiple attempts to demonstrate competency.  ``Competency'' is defined as work that demonstrates clear and complete understanding of the relevant concepts in the learning target. While the work may include a minor error, or one or two minor steps may not be well explained, otherwise the work is complete, correct, and convincing, including the proper use of notation, and appropriate explanation of meaning that uses correct terminology.  (On rare occasions, you might get a ``NY*''; such a mark means you are close to competency and that you have the option to have a short conversation with me prior to the next Checkpoint and attempt to demonstrate competency through the conversation aloud.)	
	
	\item {\bf Submission.}  When finished, use a scanning app to create digital images of your work and upload it to Blackboard under the current Checkpoint location, doing so by 1 pm.
	
\end{itemize}

\noindent \hrulefill \\

\noindent {\bf Learning Targets:} For Checkpoint \#5, there are three learning target being assessed: 

\begin{quote}
{\bf LT 5}:  I can interpret the instantaneous rate of change of a function and explain its meaning in context.  (Section 1.5)
\end{quote}

\begin{quote}
{\bf LT 6}: I can recognize and explain the relationships among the behaviors of $f$, $f'$, and $f''$ by analyzing graphical or numerical information from one or more of $f$, $f'$, and $f''$. (Section 1.6)
\end{quote}

\begin{quote}
{\bf LT 7}: I can find an equation of the tangent line to a function at a point and use the tangent line to approximate values of the function. (Section 1.8)
\end{quote}


\noindent \hrulefill \\

\vfill \hfill {\bf OVER} 

\pagebreak

%\noindent Name: \\ 

\noindent To demonstrate competency on Learning Target \#$n$, you need to correctly respond to all or nearly all of the prompts in question \#$n$.  If you have already passed a learning target on a prior Checkpoint, you don't need to re-attempt it on this one.
  
  
\begin{enumerate}
	
	\item[5.] For each of the following questions, carefully explain the meaning of the given derivative value by writing a complete sentence that discusses a particular rate of change, with appropriate units.  {\bf In addition}, for each piece of data, you should explain what you expect to happen to the value of the function as the independent variable increases by one unit.  For example, you could say something like ``at the moment $t = \ldots$, I expect that over the next minute, $\ldots$.''  
	
		\be 
			
			\item The value, $V$, of a particular car (in dollars) depends on the number of miles, $m$, the car has been driven, according to the function $V(m)$.  Explain the meaning of $V'(7000) = -1.29$ in the context of the car's value.
			
			\begin{itemize}
			
			  \item (meaning as a rate, with units)
			  
				\vspace{0.9in}
			
			  \item (what I expect to happen)
			  
			  	\vspace{0.9in}
				
			\end{itemize}
			
			\item A pie is placed in an oven to bake.  The pie's temperature, $C$, in degrees Celsius is a function of time $t$ in minutes, given by $C(t)$.  Explain the meaning of $C'(20) = 0.84$ in the context of the pie's temperature.
	
			\begin{itemize}
			
			  \item (meaning as a rate, with units)
			  
				\vspace{0.9in}
			
			  \item (what I expect to happen)
			  
			  	\vspace{0.9in}
				
			\end{itemize}
			
			\item A car's fuel consumption, $G$ (in gallons per mile), is a function of the car's speed $s$ (in miles per hour), given by $G(s)$.  Explain the meaning of $G'(55) = 0.03$ in the context of the car's fuel consumption.

			\begin{itemize}
			
			  \item (meaning as a rate, with units)
			  
				\vspace{0.9in}
			
			  \item (what I expect to happen)
			  
			  	\vspace{0.9in}
				
			\end{itemize}

		\ee

	\vfill \hfill {\bf OVER}
	
	\pagebreak
			
	\item[6.] This question asks you to use and interpret information about a function, its derivative, and its second derivative, in two different settings.
	
		\be
			
			\item On the axes provided, sketch a well-labeled possible graph of a function $y = h(x)$ that has the following properties: $h(2) = -1$, $h'(2) = -\frac{1}{2}$, $h'(x)$ is always negative (that is, $h'(x) < 0$ for all values of $x$), and $h''(x)$ is negative for $x < 2$ and $h''(x)$ is positive for $x > 2$.  Write a careful sentence to explain why you drew your graph as you did. \\

\hspace{0.45in} \scalebox{0.45}{\includegraphics{images-W25/CP5blank.jpg}} \\
\\

			\item The value of a car, $V(m)$ (in dollars), depends on the number of miles, $m$, the car has been driven.  Say we know that $V(7500) = 27350$, $V'(7500) = -1.23$, and $V''(7500) = 0.045$. \\


In everyday language, explain what we know about the behavior of the car's value at the moment when it has been driven 7500 miles.  Cite all three pieces of given information, with units.
			
			\lr \tr

		\ee

	\vfill \hfill {\bf OVER}
	
	\pagebreak	

%{\bf LT 7}: I can find an equation of the tangent line to a function at a point and use the tangent line to approximate values of the function. (Section 1.8)
		
	\item[7.] For a given function $y = f(x)$, say that we know the following information:  $f(-2) = 3.5$ and $f'(x) = \frac{1}{4}x^2 - 5$.  \\ 
	
	Throughout the following questions, proper notation is essential.
	
	\be
		\item Find a formula for $L(x)$, the tangent line approximation to $y = f(x)$ at the point $(a,f(a))$, where $a = -2$.
		
		\lr \lr
		
		\item Use the tangent line approximation you found in (a) to estimate $f(-1.9)$.  Clearly show your work using proper notation.
		
		\lr \tr \tr
		
		\item Is the estimate you found in (b) likely too large or too small?  Why?
		
		\tr
	\ee



\end{enumerate}

\end{document}
