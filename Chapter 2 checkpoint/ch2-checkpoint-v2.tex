\input{../header}
\usepackage{pgfplots}
\rhead{Your name: \rule{8cm}{0.15mm}}

\everymath{\displaystyle}
\begin{document}
%


%\onehalfspacing
\allowdisplaybreaks
%##################################################################
\section{Chapter 2 checkpoint!}

Chapter 1 scorecard:
\begin{center}
    \begin{tabular}{|m{3.75cm}|*{5}{m{2cm}|}} \hline
        Learning target: & DF1 & DF2 & DFa & DFb & AD2 \\\hline
        Your confidence level before starting (0-5): & &&&&\\\hline
        Your confidence level after the quiz (0-5): & &&&&\\\hline
        The mark you earned on this attempt: 
        & Success! \newline Try again!
        & Success! \newline Try again!
        & Success! \newline Try again!
        & Success! \newline Try again!
        & Success! \newline Try again! \\\hline

    \end{tabular}
\end{center}
Chapter 2 scorecard:
\begin{center}
    \begin{tabular}{|m{3.75cm}|*{6}{m{1.75cm}|}} \hline
        Learning target: & DF3 & DF4 & DF5 & DF6 & DF7 & DF8 \\\hline
        Your confidence level before starting (0-5): & &&&&&\\\hline
        Your confidence level after the quiz (0-5):  & &&&&&\\\hline
        The mark you earned on this attempt: 
        & Success! \newline Try again!
        & Success! \newline Try again!
        & Success! \newline Try again!
        & Success! \newline Try again!
        & Success! \newline Try again!
        & Success! \newline Try again! \\\hline

    \end{tabular}
\end{center}

Before anything else, please do the following:
\begin{itemize}
    \item Rank your confidence from 0-5 on each of the learning targets. 5 means ``I could teach a whole class about this;'' 0 means ``I am genuinely not sure I have heard these words before.''
    \item Write your name on this page and on each of the other pages of the quiz.
\end{itemize}

Then do the quiz! Some reminders:
\begin{itemize}
    \item Open notes, closed computer.
    \item If you need more room to write, use the back of the same learning target page, or ask me for some scratch paper.
    \item Read the questions carefully and make sure you're answering each part.
    \item Use good grammar for derivative problems.
    \item Show all your work and explain all your thinking!
\end{itemize}

When you are done:
\begin{itemize}
    \item Rank your confidence from 0-5 on each of the learning targets. 5 means ``I absolutely nailed that question for sure;'' 0 means ``oof, I definitely didn't get that one.''
    \item Make double sure your name is on every page, including any scratch paper.
    \item Hand in your work, separated by learning target.
\end{itemize}

Have fun and do your best! I believe in u $\heartsuit$


%%%%%%%%%%%%%%%%%%%%%%%%%%%%%%%%%%%%%%%%%%%%%%%%%%%%%%%%%
\pagebreak
%%%%%%%%%%%%%%%%%%%%%%%%%%%%%%%%%%%%%%%%%%%%%%%%%%%%%%%%%

%%% auto-generated from TBIL checkit bank
%%% https://library.tbil.org/2025/calculus/exercises/#/bank/
\section{Learning target DF3, version 2}
\providecommand{\stxKnowl}{}\renewcommand{\stxKnowl}[1]{#1}
\providecommand{\stxOuttro}{}\renewcommand{\stxOuttro}[1]{#1}
\providecommand{\stxTitle}{}\renewcommand{\stxTitle}[1]{#1}
% Comment next line to show outtros
\renewcommand{\stxOuttro}[1]{}
%%%%%%%%%%%%%%%%%%%%%%%%%%%%
\stxKnowl{
 Demonstrate and explain how to find the derivative of the following functions. Be sure to write down which derivative rules (constant multiple, sum/difference, etc.) you are using in your work.

\begin{enumerate}
\item
\stxKnowl{
 \(h(y) = -4 \, \ln\left(y\right) + 3 \, \sin\left(y\right)\) 

\stxOuttro{
 \[h'(y) = -\frac{4}{y} + 3 \, \cos\left(y\right)\] 

}
}
\vfill
\item
\stxKnowl{
 \(g(t) = \sqrt[4]{t^3} + \frac{7}{t^{4}}\)

\stxOuttro{
 \[g'(t) = \frac{3}{4 \, t^{\frac{1}{4}}} - \frac{28}{t^{5}}\] 

}
}
\vfill
\item
\stxKnowl{
 \(f(x) = -x^{5} - 4 \, x^{4} + 5 \, x - 4\)

\stxOuttro{
 \[f'(x) = -5 \, x^{4} - 16 \, x^{3} + 5\] 

}
}
\vfill

\end{enumerate}
}

%%%%%%%%%%%%%%%%%%%%%%%%%%%%%%%%%%%%%%%%%%%%%%%%%%%%%%%%%
\pagebreak
%%%%%%%%%%%%%%%%%%%%%%%%%%%%%%%%%%%%%%%%%%%%%%%%%%%%%%%%%

\section{Learning target DF4, version 2}
\providecommand{\stxKnowl}{}\renewcommand{\stxKnowl}[1]{#1}
\providecommand{\stxOuttro}{}\renewcommand{\stxOuttro}[1]{#1}
\providecommand{\stxTitle}{}\renewcommand{\stxTitle}[1]{#1}
% Comment next line to show outtros
\renewcommand{\stxOuttro}[1]{}
%%%%%%%%%%%%%%%%%%%%%%%%%%%%
\stxKnowl{
 Demonstrate and explain how to find the derivative of the following functions. Be sure to write down which derivative rules (product, quotient, sum and difference, etc.) you are using.

\begin{enumerate}
\item
\stxKnowl{
\(\renewcommand{\log}{\ln} h(w)= -\frac{\cos\left(w\right)}{6 \, w^{2} + 5 \, w + 4}\)

\stxOuttro{
\[\renewcommand{\log}{\ln} g' (x)= -\frac{2 \, {\left(2 \, x + 3\right)} \log\left(x\right)}{{\left(2 \, x^{2} + 6 \, x - 3\right)}^{2}} + \frac{1}{{\left(2 \, x^{2} + 6 \, x - 3\right)} x}\]

}
}
\vfill
\item
\stxKnowl{
\( \renewcommand{\log}{\ln} g(w)= \frac{3 \, w^{2} + 5 \, w - 1}{w^{8}}\)

\stxOuttro{
\[\renewcommand{\log}{\ln} f' (x)= 2 \, {\left(x - 1\right)} \cos\left(x\right) - {\left(x^{2} - 2 \, x - 4\right)} \sin\left(x\right)\]

}
}
\vfill
\item
\stxKnowl{
\(\renewcommand{\log}{\ln} f(w)= -{\left(4 \, w^{2} + 6 \, w + 1\right)} \log\left(w\right)\)

\stxOuttro{
\[\renewcommand{\log}{\ln} h' (t)= \frac{3 \, t^{2} + 4 \, t + 9}{t^{4}}\]

}
}
\vfill
\end{enumerate}
}

%%%%%%%%%%%%%%%%%%%%%%%%%%%%%%%%%%%%%%%%%%%%%%%%%%%%%%%%%
\pagebreak
%%%%%%%%%%%%%%%%%%%%%%%%%%%%%%%%%%%%%%%%%%%%%%%%%%%%%%%%%

\section{Learning target DF5, version 2}
\providecommand{\stxKnowl}{}\renewcommand{\stxKnowl}[1]{#1}
\providecommand{\stxOuttro}{}\renewcommand{\stxOuttro}[1]{#1}
\providecommand{\stxTitle}{}\renewcommand{\stxTitle}[1]{#1}
% Comment next line to show outtros
\renewcommand{\stxOuttro}[1]{}
%%%%%%%%%%%%%%%%%%%%%%%%%%%%
\stxKnowl{
 Demonstrate and explain how to find the derivative of the following functions. Be sure to write down which derivative rules (product, quotient, sum and difference, etc.) you are using. 

\begin{enumerate}
\item
\stxKnowl{
\(h(x)= -9 \, \cos\left(-5 \, x^{2} + 3 \, x + 4\right)\)

\stxOuttro{
\[h' (t)= 64 \, {\left(t + e^{t} + 2\right)}^{3} {\left(e^{t} + 1\right)}\]

}
}
\vfill
\item
\stxKnowl{
\(f(y)= -9 \, \sin\left(y^{\frac{7}{2}}\right)\)

\stxOuttro{
\[f' (w)= \frac{35}{3} \, w^{\frac{4}{3}} \cos\left(w^{\frac{7}{3}}\right)\]

}
}
\vfill
\item
\stxKnowl{
\(g(t)= -9 \, \Big(\sin\left(t\right)\Big)^{\frac{7}{2}}\)

\stxOuttro{
\[g' (x)= \frac{35}{3} \, \cos\left(x\right) \sin\left(x\right)^{\frac{4}{3}}\]

}
}
\vfill
\item \(k(w)= {\left(3 \, w + e^{w} - 1\right)}^{4}\)
\vfill
\end{enumerate}
}

%%%%%%%%%%%%%%%%%%%%%%%%%%%%%%%%%%%%%%%%%%%%%%%%%%%%%%%%%
\pagebreak
%%%%%%%%%%%%%%%%%%%%%%%%%%%%%%%%%%%%%%%%%%%%%%%%%%%%%%%%%

\section{Learning target DF6, version 2}
\providecommand{\stxKnowl}{}\renewcommand{\stxKnowl}[1]{#1}
\providecommand{\stxOuttro}{}\renewcommand{\stxOuttro}[1]{#1}
\providecommand{\stxTitle}{}\renewcommand{\stxTitle}[1]{#1}
% Comment next line to show outtros
\renewcommand{\stxOuttro}[1]{}
%%%%%%%%%%%%%%%%%%%%%%%%%%%%
\stxKnowl{
 Demonstrate and explain how to find the derivative of the following functions. Be sure to write down which derivative rules (constant multiple, sum and difference, etc.) you are using.

\begin{enumerate}
\item
\stxKnowl{
\(f(w) = \left( \frac{5 \, w^{6} + 1}{5 \, w^{6} + 2} \right)^{ 6 }\)

\stxOuttro{
\[g ' (x) = \frac{4 \, x^{3} \sin\left(-2 \, x^{4} + 5\right)}{\sqrt{\cos\left(-2 \, x^{4} + 5\right)}}\]

}
}
\vfill
\item
\stxKnowl{
\(g(x) = {\left(5 \, x^{5} - 2 \, x^{3}\right)}^{6} \sqrt{x}\)

\stxOuttro{
\[f ' (w) = 4 \left( \frac{6 \, {\left(w^{6} - 1\right)}}{5 \, w + 3} \right)^{ 3 } \left( \frac{36 \, w^{5}}{5 \, w + 3} - \frac{30 \, {\left(w^{6} - 1\right)}}{{\left(5 \, w + 3\right)}^{2}} \right)\]

}
}
\vfill
\item
\stxKnowl{
\(h(y) = \sqrt{\sin\left(-2 \, y^{4} + 4\right)}\)

\stxOuttro{
\[h ' (y) = 4 \, {\left(3 \, y^{2} + 7 \, y\right)}^{3} {\left(6 \, y + 7\right)} y^{\frac{1}{4}} + \frac{{\left(3 \, y^{2} + 7 \, y\right)}^{4}}{4 \, y^{\frac{3}{4}}}\]

}
}
\vfill
\end{enumerate}
}

%%%%%%%%%%%%%%%%%%%%%%%%%%%%%%%%%%%%%%%%%%%%%%%%%%%%%%%%%
\pagebreak
%%%%%%%%%%%%%%%%%%%%%%%%%%%%%%%%%%%%%%%%%%%%%%%%%%%%%%%%%

\section{Learning target DF7, version 2}
\providecommand{\stxKnowl}{}\renewcommand{\stxKnowl}[1]{#1}
\providecommand{\stxOuttro}{}\renewcommand{\stxOuttro}[1]{#1}
\providecommand{\stxTitle}{}\renewcommand{\stxTitle}[1]{#1}
% Comment next line to show outtros
\renewcommand{\stxOuttro}[1]{}
%%%%%%%%%%%%%%%%%%%%%%%%%%%%
\stxKnowl{

\begin{enumerate}
\item
\stxKnowl{
Use implicit differentiation to find \(\frac{dy}{dx}\), aka \(y'\), for the equation \(5 \, x^{5} - 2 \, \sin\left(y\right) = -8 \, y^{3} - 7\).

\stxOuttro{
\[\frac{dy}{dx}=\frac{7 \, x^{3}}{3 \, y^{3} + \sin\left(y\right)}\]

}
}
\vfill
\item
\stxKnowl{Use implicit differentiation to find \(\frac{dy}{dx}\), aka \(y'\), for the equation \(0 = -y\, \cos\left(x\right) - 2 \, e^{x}\).

\stxOuttro{
\(\frac{dy}{dx}=\dfrac{3x^2-9y}{9x+3y^2}\), so the slope at \((-2,4)\) is \(-0.8\), and the equation of the tangent line is \(L(x) = -0.8(x+2) +4\). Since \(L(-2.5) = 4.4\), the point \((-2.5, 4.4)\) is pretty close to being on the original function.

}
}
\vfill
\end{enumerate}
}



\end{document}