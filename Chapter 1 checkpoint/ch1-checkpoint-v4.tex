\input{../header}
\usepackage{pgfplots}
\rhead{Your name: \rule{8cm}{0.15mm}}

\begin{document}
%


%\onehalfspacing
\allowdisplaybreaks
%##################################################################
\section{Learning target DF1, version 4}
A water balloon is tossed vertically in the air from a window. The balloon's height (measured in feet) at time $t$ (measured in seconds after being launched) is given by \(s(t) = -16t^2 + 16t + 32\); \\
a graph of this function is provided below.

\includegraphics[width=\textwidth]{../images/DF1-v4.png}

\begin{enumerate}[(a)]
    \item Compute $s(1)$. Show your work.
    \vfill
    \item On the graph above, carefully sketch the \textit{tangent} line to $s(t)$ at $t=1$.
    \item On the graph above, carefully sketch a \textit{secant} line through the point on the graph at $t=1$ and a second nearby point.
    \item Compute the slope of your secant line. Show your work. 

    \vspace{-0.7em}
    {\tiny Hints: rise over run; one of your two $y$-values is $s(1)$, which you computed in part (a).}
    \vfill
    \item Use shortcut rules to find $s'(t)$, and compute $s'(1)$; show your work. Compare your result here to your result in part (d). Do these two numbers make sense together? Why?
    \vfill
\end{enumerate}

%%%%%%%%%%%%%%%%%%%%%%%%%%%%%%%%%%%%%%%%%%%%%%%%%%%%%%%%%
\pagebreak
%%%%%%%%%%%%%%%%%%%%%%%%%%%%%%%%%%%%%%%%%%%%%%%%%%%%%%%%%

\section{Learning target DF2, version 4}

Suppose that  \(f(x)=4x^2-3x+2\). Use the limit definition of the derivative to find $f'(x)$.

%%%%%%%%%%%%%%%%%%%%%%%%%%%%%%%%%%%%%%%%%%%%%%%%%%%%%%%%%
\pagebreak
%%%%%%%%%%%%%%%%%%%%%%%%%%%%%%%%%%%%%%%%%%%%%%%%%%%%%%%%%

\section{Learning target DFa, version 4}

In order to prepare for the upcoming ski season, a local resort is making snow on some of its lower trails. The snow depth, in inches, at time $t$ is $D(t)$, where $t$ is the number of days we are into December (so, for instance, $t=9$ is December 9). 

Write a sentence explaining what each of these equations means about the snow depth at the resort. \textbf{Give units} to every number that you write down; \textbf{don't say ``per'' and don't say ``rate''.}
\begin{enumerate}[(a)]
\item $D(15) = 28$ 
\vfill
\item $D'(15) = 1.4$
\vfill
\end{enumerate} 

%%%%%%%%%%%%%%%%%%%%%%%%%%%%%%%%%%%%%%%%%%%%%%%%%%%%%%%%%
\pagebreak
%%%%%%%%%%%%%%%%%%%%%%%%%%%%%%%%%%%%%%%%%%%%%%%%%%%%%%%%%

\section{Learning target DFb, version 4}

Here is the graph of some wacky function $q(t)$:

\begin{center}
    \includegraphics[width=0.9\textwidth]{../images/DFb-v4.png}    
\end{center}


Sketch the graph of $q'(t)$ on the blank axes below.

\begin{center}
    \includegraphics[width=0.9\textwidth]{../images/DFb-v4-blank.png}    
\end{center}

%%%%%%%%%%%%%%%%%%%%%%%%%%%%%%%%%%%%%%%%%%%%%%%%%%%%%%%%%
\pagebreak
%%%%%%%%%%%%%%%%%%%%%%%%%%%%%%%%%%%%%%%%%%%%%%%%%%%%%%%%%

\section{Learning target AD2, version 4}

Part of the graph of this function $f(x)$ has been erased. 
We'll use linear approximation to recover the values.

\includegraphics[width=\textwidth]{../images/linapprox.png}

\begin{enumerate}[(a)]
\item Draw the tangent line to this function at $x = 1$.
\item Suppose we know that $f'(1) = 0.65$. Use point-slope form to write down an equation for $L(x)$, the tangent line to this function at $x=1$.
\vfill
\item Use your equation for $L(x)$ to estimate $f(1.4)$.
\vfill
\item Do you think it is more likely that this is an 
overestimate or an underestimate? How come?
\vfill
\end{enumerate}


\end{document}